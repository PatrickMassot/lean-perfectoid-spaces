\documentclass{amsart}
\usepackage{amsmath}
\usepackage{amsthm}
%\usepackage{a4wide}
%\usepackage{enumerate}

\newcommand{\Q}{\mathbb{Q}}
\newcommand{\R}{\mathbb{R}}
\newcommand{\Z}{\mathbb{Z}}
\DeclareMathOperator{\Spa}{Spa}
\DeclareMathOperator{\Spv}{Spv}

\theoremstyle{plain}
\newtheorem{theorem}{Theorem}
\newtheorem{lemma}[theorem]{Lemma}
\newtheorem{corollary}[theorem]{Corollary}
\newtheorem{proposition}[theorem]{Proposition}
\theoremstyle{remark}
\newtheorem{remark}[theorem]{Remark}
\newtheorem*{remarkn}{Remark}

\title{Overview of definition of a perfectoid space.}
\begin{document}

\maketitle

This document was written on and shortly after 17th Sept 2018.

\section{Huber rings.}

A \emph{Huber ring} is a topological ring $A$ satisfying the following additional axiom: there exists an open subring $A_0$ containing a finitely-generated ideal $J$ such that the induced topology on $A_0$ is the $J$-adic one. Note that $A_0$ is not part of the data.

Note: this terminology is due to Scholze. The older terminology for this notion is an ``$f$-adic ring''.

A basic example is the $p$-adic numbers $\Q_p$ with its usual topology, with subring $\Z_p$ having topology generated by the ideal $(p)$. Note that the subring $\Q_p$ won't work -- the topology on $\Q_p$ is ``$p$-adic'' in some sense, but it is not the topology generated by the ideal $(p)$ of $\Q_p$, because this ideal is all of $\Q_p$.

Equivalent definitions of a Huber ring, and basic theorems about these rings, can be found in Proposition/Definition 6.1 of \cite{wedhorn} and the lemmas following it.

The notion of a Huber ring has been formalised in the perfectoid project, in the file {\tt Huber\_ring.lean}.

\section{Huber pairs.}

A \emph{Huber pair} is a pair $(A,A^+)$ consisting of a Huber ring $A$ and an open integrally-closed subring of power-bounded elements. Again this is Scholze's terminology -- these used to be called ``affinoid rings''. 

Details of what these words mean, and some basic properties of Huber pairs, can be found in Definition~7.14 of~\cite{wedhorn}.

An example of a Huber pair is the pair $(\Q_p,\Z_p)$. Another example is $(\Q_p[T],\Z_p[T])$ with $\Z_p[T]$ given the $p$-adic topology (that is, a polynomial is small if and only if all its coefficents are $p$-adically small -- we don't mind if the degree is big or if the constant coefficient is non-zero).

The notion of a Huber pair has been formalised in the perfectoid project, in the file {\tt Huber\_pair.lean}.

\section{The adic spectrum of a Huber pair.}

The \emph{adic spectrum} $\Spa(A)$ of a Huber pair $(A,A^+)$ (note abuse of notation; $A^+$ is often not mentioned even though it is not uniquely determined by $A$) is quite an elaborate object.

As a set (or a type, if you prefer), it is just the equivalence classes of continuous valuations on $A$ which are bounded by 1 on $A^+$. Valuations are defined in Definition~1.22 of Wedhorn, the equivalence relation is 1.27, continuous valuations are definition 7.7. All of these notions have been formalised in the perfectoid project, and the type {\tt Spa A} for $A$ a Huber pair is defined in {\tt Spa.lean}. An implementation remark: because, formally, the valuations equivalent to a given valuation form a proper class (for trivial reasons -- one can just enlarge the target space of the valuation to be as large as one likes as there is no surjectivity condition) there are universe issues. The design decision made is that if a ring {\tt R} lives in universe {\tt u} then a valuation on {\tt R} can live in another universe {\tt v}, however when forming {\tt Spa A} we restrict to valuations taking values in groups which live in the same universe as {\tt A}, and prove a lemma saying that for any valuation on {\tt A} there is an equivalent valuation taking values in a group which lives in the same universe as {\tt A}. 

This set comes with a bunch of extra structures, not all of which are currently in Lean.

1) It has the structure of a topological space, and amongst the open subsets are a special kind of open subsets called \emph{rational subsets}. The definition of the topological space structure on $\Spa(A)$ is that it is a subset of the space $\Spv(A)$ of all valuations on $A$, which has a topology defined in Definition 4.1 of~\cite{wedhorn}; $\Spa(A)$ gets the subspace topology. The definition of a rational subset is Definition 7.29 of~\cite{wedhorn}. Apparently we still have not proved that rational subsets are open -- this is sorried in the perfectoid project.

2) The topological space $\Spa(A)$ has a presheaf of complete topological rings on it. The presheaf is first defined on the rational subsets. To make the definition on the rational subsets one has to have localisations of rings (which mathlib has), and completions of topological rings, which it does not yet quite have (but we are nearly there, thanks to Patrick Massot's valiant efforts). The definition of this presheaf in full is in section 8.1 of~\cite{wedhorn}. To extend from rational subsets to all open subsets we need to take a projective limit; the definition of the presheaf on an arbitrary open is the projective limit of the presheaf on all the rational subsets contained in the open. If the open is itself a rational subset it's hence trivial that we didn't just change the definition (up to canonical isomorphism). 

3) Furthermore, the stalk of (the underlying presheaf of rings associated to) this presheaf of topological rings at a point $x\in\Spa(A)$, can be shown to be a local ring, and furthermore the equivalence class of valuations associated to $x$ gives rise to an equivalence class of valuations on this loal ring, whose support is the maximal ideal; this is remark and definition 8.5 of~\cite{wedhorn}. This stuff is not yet in Lean but should not be too hard (famous last words).

{\bf Current status}: We need to be able to complete a topological ring and prove that the completion has the universal property stated in 5.32 of \cite{wedhorn}. This is blocking (2). After that I am hoping (3) should not be too hard, as we have a bunch of stuff on valuations, but as far as I know nobody has thought about this.

\section{Adic spaces}

An adic space is a topological space with a \emph{sheaf} of topological rings, such that the stalks (in the underlying sheaf of rings) are local, and each of these is equipped with a valuation whose support is the maximal ideal, and which has an open cover by subsets each isomorphic to $Spa(A)$ for some Huber pair $A$. Just o be completely clear -- the isomorphism takes place in a category called $V^{pre}$ by Wedhorn, but every mathematician knows what we mean here. Key definitions in \cite{wedhorn}: $V^{pre}$ is on p76, $\mathcal{V}$ on p80, adic space is definition 8.21.

\section{Perfectoid spaces}

A Huber ring is a \emph{Tate ring} if it has a topologically nilpotent unit. This is definition~6.10 of~\cite{wedhorn}. It's also in Scholze in section 3 I guess.


A perfectoid space is just an adic space with some extra property. All this is already in Lean.

\section{Refs}

Wedhorn adic spaces
Scholze etale cohomology of diamonds
\end{document}
