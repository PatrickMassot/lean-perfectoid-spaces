\documentclass{amsart}
\usepackage{amsmath}
\usepackage{amsthm}
%\usepackage{a4wide}
%\usepackage{enumerate}

\newcommand{\R}{\mathbb{R}}
\newcommand{\Z}{\mathbb{Z}}

\theoremstyle{plain}
\newtheorem{theorem}{Theorem}
\newtheorem{lemma}[theorem]{Lemma}
\newtheorem{corollary}[theorem]{Corollary}
\newtheorem{proposition}[theorem]{Proposition}
\theoremstyle{remark}
\newtheorem{remark}[theorem]{Remark}
\newtheorem*{remarkn}{Remark}

\begin{document}

\section{Notes on {\tt valuations.lean}}

We define {\tt linear\_ordered\_comm\_group} and {\tt linear\_ordered\_comm\_monoid} to be a commutative group / monoid with a linear order satisfying $a\leq b\implies ca \leq cb$. Within the namespace {\tt linear\_ordered\_comm\_group} we prove some basic lemmas.

We then define {\tt is\_convex}, a property of subgroups of a linearly ordered commutative group. A subgroup $H$ is convex if whenever $x\leq z\leq y$ and $x,y\in H$ we have $z\in H$. An example to keep in mind is that if $\mathbb{R}^2$ has the lexicographic order $(a,b)< (c,d)\iff a<c \lor (a=c \land b<d)$ then the subgroup $$\{0\}\times\mathbb{R}$$ is convex. The kernel of an order-preserving group homomorphism is shown to be convex (an example would be the projection onto the first factor in the $\mathbb{R}^2$ example above).

The height of a linearly ordered commutative group is the number of proper convex subgroups. For example one can check that $\mathbb{R}^2$ above has height~2.

The construction of a totally ordered monoid $\Gamma\cup\{0\}$ from a totally ordered group $\Gamma$ is done using the {\tt option} function -- the ``extra'' element {\tt none} is identified with zero. It is proved that this extension of a linearly ordered group is a linearly ordered monoid. All of this happens within the {\tt linear\_ordered\_comm\_group} namespace.

We then define a valuation on a commutative ring, taking values in the monoid associated to a linearly ordered commutative group. Within the {\tt is\_valuation} namespace we prove some basic results about valuations. For some reason {\tt is\_valuation} is a typeclass. Some basic lemmas about valuations are proved, including the fact that the support of a valuation is a prime ideal.
\end{document}
